{\bf Belangrijke begrippen analyse 2 week 18.}
\vskip 8pt
\begin{description}
\item[limiet] Als $a$ een verdichtingspunt is van $A$, het domein van
de functie $f$, dan betekent $\lim_{x\to a} f(x)=L$:
voor elke $\epsilon>0$ is er een $\delta>0$ zodat als 
$x\in A\setminus\{a\}$ en $|x-a|<\delta$, 
dan $|f(x)-L|<\epsilon$.
\item[hoogtelijnen] Voor level curves: aarzel niet Mathematica te gebruiken.
\item[tip limieten] Voor limieten: probeer zoveel mogelijk de insluitstelling
te gebruiken. Ook Taylorreeksen zijn vaak nuttig.
\item[Som Kosmala] 10.2.9: onderscheid de gevallen $x\ne y$ en $x=y$.
\end{description}
\vskip 8pt
{\bf Belangrijke begrippen analyse 2 week 19.}
\begin{description}
\item[Uniform continu] Voor elke $\epsilon>0$ is er
een $\delta>0$ zodanig dat voor elke $a$ en $b$ in het domein van $f$ 
geldt $|f(a)-f(b)|<\epsilon$ als $|a-b|<\delta$.
\item[Banach] Als $f\,:\,D\to D(\subset \mathbb R^d)$ een contractie
is en $D$ gesloten, dan is er een uniek vast punt $p$ (dus waarvoor $f(p)=p$), en voor elke
$x\in D$ convergeert de rij $x,f(x),f(f(x)),\dots$ naar dit punt $p$. \\
$f$ heet contractie op $D$ als er een $q<1$ is met $|f(x)-f(y)|\le q|x-y|$
voor alle $x,y\in D$.
\end{description}
\vskip 8pt
{\bf Belangrijke begrippen analyse 2 week 20.}
\begin{description}
\item[Parti\"ele afgeleiden] $\ds \partial_xf(a,b)=\lim_{h\to 0}\frac{f(a+h,b)-f(a,b)}h$.
Iets dergelijks voor $\partial_y$ en voor functies van (nog) meer variabelen.
Beetje verwarrend: met $\partial_{xy}f$ of ook $f_{xy}$ bedoelen we $\partial_y(\partial_xf)$,
we differentieren hier dus {\em eerst} naar $x$ en dan naar $y$ (voor {\em fatsoenlijke}
functies maakt de volgorde overigens niet uit).
\item[Differentieerbaar] $f\,:\,\mathbb R^2\to\mathbb R$ is differentieerbaar in $(a,b)$
als er getallen $m_1$ en $m_2$ zijn met 
$f(a+h,b+k)=f(a,b)+m_1h+m_2k+\epsilon\sqrt{h^2+k^2}$ waarbij $\epsilon\to0$ als $(h,k)\to(0,0)$.
hier zijn $m_1$ en $m_2$ de parti\"ele afgeleiden $\partial_xf(a,b)$ en $\partial_y$. 
Het beginstuk heet de linearisering van $f$ rond $(a,b)$ en geeft de vergelijking van het
raakvlak. Iets dergelijks voor meer variabelen.
\end{description}