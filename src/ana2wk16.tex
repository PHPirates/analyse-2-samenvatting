{\bf Belangrijke begrippen analyse 2 week 15.}
\vskip 8pt
\begin{description}
\item[bol, ball] $B(a,r)=\{x\in \mathbb R^d\,|\,|x-a|<r\}$, de open bol rond $a$ met
straal $r$ (in dimensie $d$). $|x-a|$ is de afstand tussen de punten $a$ en $x$,
als $d=1$ is dit hetzelfde als de absolute waarde. Kosmala zou schrijven
$\|\vec x-\vec a\|$.
\item[inwendig punt, interior point] $a\in A$ heet inwendig punt als er een $r>0$ is
met $B(a,r)\subseteq A$ (voor mij zijn $\subseteq$ en $\subset$ gelijkwaardig). Een
inwendig punt behoort noodzakelijk tot de verzameling.
\item[open, inwendige, interior]
$\inw~A$, het inwendige van $A$ is de verzameling inwendige punten van $A$. 
$A$ heet open als
$A=\inw~A$. 
\item[rand, boundary] Punt $a$ heet een randpunt van $A$ als voor elke $r>0$ de bol $B(a,r)$
zowel punten in $A$ als punten niet in $A$ bevat, de rand van $A$ geven we aan met $\partial A$. 
Een randpunt hoeft niet tot de verzameling te behoren.  
\end{description}
{\bf Belangrijke begrippen analyse 2 week 16.}
\vskip 8pt
\begin{description}
\item[verdichtingspunt, accumulation point]
Punt $a$ is {\em verdichtingspunt} van verzameling $A$ als voor alle $r>0$, $B(a,r)$ een punt
van $A$ bevat {\em verschillend} van $a$. $A'$ is de collectie verdichtingspunten. Een 
verdichtingspunt hoeft niet tot de verzameling te behoren.
\item[afsluiting, closure]
De afsluiting $\overline A$ is $A$ samen met zijn verdichtingspunten, het is ook $A$ samen
met zijn rand $\partial A$. $A$ heet gesloten (closed) als hij gelijk is aan zijn afsluiting, dus
als hij al zijn verdichtingspunten bevat, dus als de rand van $A$ in $A$ zit. $A$ is altijd bevat
in zijn afsluiting.
\item[rijen en limieten]
$x^{(n)}\to a$ betekent $\lim_{n\to\infty} x^{(n)}=a$, oftewel:\\ 
voor elke $\epsilon>0$ geldt dat de bol
$B(a,\epsilon)$ voor zekere $N$ alle $x^{(n)}$ bevat waarvoor $n>N$. Oftewel:\\
$\forall\epsilon>0: ~\exists N\in \mathbb N: \forall n>N: |x^{(n)}-a|<\epsilon$. \\
Als de rij $x^{(n)}$ een limiet heeft dan heet de rij {\em convergent}.
\item[infimum en supremum]
Als $A\subset \mathbb R$ niet leeg is en van onderen begrensd, 
dan betekent $m=\inf A$ dat \\
(i) $m\le a$ voor alle $a\in A$; \\
(ii) $m$ is maximaal met betrekking tot eigenschap (i), 
oftewel:\\ 
voor elke $\epsilon>0$ is er een $a\in A$ met $a<m+\epsilon$.\\
Conventies: Als $A$ niet naar beneden begrensd is zeggen ook wel
$\inf A=-\infty$. Als $A=\emptyset$ dan zeggen we ook wel
$\inf A=\infty$.\\ 
Verzin zelf wat $M=\sup A$ betekent.
\end{description}
{\bf Belangrijke begrippen analyse 2 week 17.}
\vskip 8pt
\begin{description}
\item[begrensd, bounded] $A$ heet begrensd als er een $r$ is met $|a|<r$ voor alle
$a\in A$ (dus $A\subseteq B(0,r)$).
\item[(open) overdekking, cover] Een collectie (open) verzamelingen $A_i$, $i\in I$ ($I$ staat 
hier voor een verzameling indices) heet (open) overdekking van $A$ als $A\subseteq \bigcup_{i\in I} A_i$,
dus voor elke $a\in A$ is er een $i\in I$ met $a\in A_i$. Als $J$ een eindige deelverzameling is van
$I$ en als ook $A\subseteq \bigcup_{j\in J} A_j$, dan is dit een eindige deeloverdekking. 
\item[compact] $A$ heet compact als {\em elke} open overdekking van $A$ een eindige deeloverdekking
heeft. Een compacte verzamelingen (in $\mathbb R^d$) is gesloten {\em en} begrensd, dit is `eenvoudig',
omgekeerd is elke gesloten begrensde verzameling compact, dit is niet eenvoudig. `Veel' uitspraken die
waar zijn voor eindige verzamelingen, en onwaar voor willekeurige oneindige verzamelingen, zijn wel waar
voor compacte verzamelingen. 
\end{description}