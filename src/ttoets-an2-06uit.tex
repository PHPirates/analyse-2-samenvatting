\documentclass[12pt,dutch]{article}
\usepackage{amssymb,amsmath}
\newcommand\inw{\rm int}
\newcommand\ds{\displaystyle}
\begin{document}
{\bf Uitwerking tussentoets Analyse 2, mei 2016.}
\vskip 6pt
\begin{description}
\item[1.] Zij $A\subset \mathbb R^d$ en $a$ een inwendig punt van $A$. Laat zien:
Voor elke rij $(x_n)$ in $\mathbb R^d$ die naar $a$
convergeert is er een $n_0\in\mathbb N$ zodanig dat $x_n \in A$ voor alle
$n \ge n_0$.\hfill {\sc 4 punten}
\item[Antw.] Als $a$ inwendig punt is, dan is er een $r>0$ met $B(a,r)\subset A$.
Als $(x_n)$ naar $a$ convergeert dan geldt voor alle $\epsilon>0$, dus ook voor
de eerder gevonden $r$ dat er een $n_0$ is met $|x_n-a|<\epsilon$ als $n\ge n_0$.
Dus voor de $n_0$ die hoort bij $r$ geldt $x_n\in B(a,r)\subset A$ als $n\ge n_0$.
\item[2.] Is de volgende bewering waar? (Geef een bewijs of een tegenvoorbeeld.)\\
``Voor willekeurige deelverzamelingen $A,B \subset \mathbb R^d$ geldt
\[
\overline{A\cup B} = \overline A\cup \overline B.~"
\]
\hfill {\sc 5 punten}
\item[Antw.] We gebruiken (bijvoorbeeld) het feit dat de afsluiting van
$C$ gelijk is aan $C$ samen met de verdichtingspunten. Elk verdichtingspunt
$x$ van $A$ is uiteraard ook verdichtingspunt van $A\cup B$, een rij in $A$ die
naar $x$ convergeert is bevat in $A\cup B$, evenzo is elk verdichtingspunt
van $B$ verdichtingspunt van $A\cup B$, dus rechts is bevat in links.
Omgekeerd als $x$ verdichtingspunt is van $A\cup B$, dan is er een rij in $A\cup B$
die naar $x$ convergeert, en deze rij bevat een oneindige deelrij die hetzij in $A$
hetzij in $B$ bevat is (allebei kan natuurlijk ook). Dus links is bevat in rechts. 
\item[3.] Ga na of de volgende limieten bestaan en zo ja, bepaal ze:
\[
\mbox{\bf a)} \lim_{(x,y)\to(0,0)}\frac{e^{xy}-1}{x^2 + y^2}~,~~~~
\mbox{\bf b)} \lim_{(x,y)\to(0,0)}\frac{(e^{xy}-1)^2}{x^2+y^2}~.
\]
\hfill 3+3 {\sc punten}
\item[Antw.] Het best gebruiken we de standaardlimiet $\ds \lim_{t\to 0}\frac{e^t-1}t=\left.\frac{de^t}{dt}\right|_{t=0}=1$. 
\[
\lim\frac{e^{xy}-1}{xy}\cdot\frac{xy}{x^2+y^2}
\]
bestaat niet, bijvoorbeeld omdat dit langs $x=0$ naar nul gaat, en langs $y=x$ naar $\frac12$.
\[
\lim\frac{(e^{xy}-1)^2}{(xy)^2}\cdot\frac{x^2y^2}{x^2+y^2}=0,
\]
want $0\le x^2y^2/(x^2+y^2)\le y^2$ en dit gaat naar nul met de insluitstelling.
\item[4.] Zij $f\, :\, \mathbb R^2 \to \mathbb R$ gegeven door
\[
f(x,y) =\left\{ 
\begin{array}{rcc}
-x^2 & \mbox{als} & y\ge0~,\\
x^2  & \mbox{als} & y<0~.
\end{array}
\right.
\]
Ga na in welke punten de functie $f$ continu/discontinu is. (Beredeneer je antwoord en let op de
volledigheid.) \hfill 5 {\sc punten}
\item[Antw.] Voor punten $p=(x,y)$ met $y>0$ geldt dat er een bol $B(p,r)$ is waar de functie
gedefini\"eerd is als $-x^2$, dus hier is $f$ zeker continu, iets dergelijks geldt als $y<0$.
In de buurt van het punt $(x,0)$ worden zowel waarden in de buurt van $x^2$ als in de buurt van
$-x^2$ aangenomen. Als $x\ne0$ dan is $x^2\ne -x^2$ dus hier is de functie discontinu.
In de buurt van de oorsprong geldt dat $|f(x,y)|\le x^2$ met limiet $0$, dus in het punt $(0,0)$
is de functie continu. 
\end{description}
\end{document} 