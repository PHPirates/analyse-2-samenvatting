\documentclass{2wa40summary}

\begin{document}
	\section{Beste allemaal}
	
	Mijn complimenten -- dit is de eerste keer dat ``m'n'' studenten alle lecture notes in \LaTeX typen.
	Prima job!
	
	\ \\
	Ik heb leesbare tekst toegevoegd (in \blu{blauw}).
	
	\ \\
	Ik heb in het \verb$.tex$ bestand definities toegevoegd -- geen aktie noodzakelijk -- tussen
	``start nieuwe macros'' en ``einde nieuwe macros'' die het
	typewerk sterk kunnen vergemakkelijken. ``Makkelijk'' is smaakafhankelijk.
	
	\ \\
	Ik stel voor -- maar dat hoeft vanzelfsprekend niet -- dat ``jullie'' de typografie aanpassen aan wat in de Analyse/Calculus gebruikelijk is -- een deel heb ik al als voorbeeld voorgedaan in de brontekst van dit bestand:
	\begin{itemize}
		\item In plaats van de letter x, y en z \textbf{die vectoren zijn in dit} \verb$.tex$ \textbf{bestand}, type
		\verb$\x$, \verb$\y$ en \verb$\z$: $\x, \y, \z$
		\item Voor alle vectoren zoals $\mathbf{a}$, $\mathbf{b}$ etc: Type \verb$\a$, \verb$\b$, etc
		\item Ditto voor vectorwaardige functies \verb$\f\colon\reals^d\mapsto\reals$: $\f\colon\reals^d\mapsto\reals$: Type \verb$\f$
		\item In plaats van de letter 0 type voor nulvectoren \verb$\zero$: $\zero$
		\item In plaats van \verb$|x|$ type \verb$\abs{x}$: $\abs{x}$
		\item In plaats van \verb$|\x|$ type \verb$\abs{\x}$: $\abs{\x}$
		\item In plaats van \verb$x^{(n)}$ type \verb$x\up{n}$: $x\up{n}$
		\item In plaats van \verb$\x^{(n)}$ type \verb$\x\up{n}$: $\x\up{n}$
		\item In plaats van \verb$\Leftrightarrow$ type \verb$\iff$: $\iff$
		\item In plaats van \verb$\Rightarrow$ type \verb$\implies$: $\implies$
		\item In plaats van \verb$\mathbb{R}$ type \verb$\reals$: $\reals$ (heb ik al vervangen)
		\item In plaats van \verb$\mathbb{N}$ type \verb$\naturals$: $\naturals$ (heb ik al vervangen)
		\item In plaats van \verb$:$ type \verb$\colon$: $x\colon y$
		\item Voor verzamelingen type bv \verb$\set{1,2,3}$: $\set{1,2,3}$
		\item Voor rode tekst gebruik \verb$\red{word 1 and 2}$ type \verb$\red{word 1 and 2}$: \red{word 1 and 2}
		\item Voor groene tekst gebruik \verb$\grn{word 1 and 2}$ type \verb$\grn{word 1 and 2}$: \grn{word 1 and 2}
		\item Voor blauwe tekst gebruik \verb$\blu{word 1 and 2}$ type \verb$\blu{word 1 and 2}$: \blu{word 1 and 2}
		\item \blu{Het} \verb$\blu{...}$ \blu{kleuren commando kan geen} \verb#\verb$...$# \blu{constructie bevatten.}
		\item De langere variant \verb$\{\color{blue}...}$ kan wel \verb#\verb$...$# in de \dots bevatten!
		\item De langere variant \verb$\{\color{red}...}$ kan wel \verb#\verb$...$# in de \dots bevatten!
		\item De langere variant \verb$\{\color{green}...}$ kan wel \verb#\verb$...$# in de \dots bevatten!
		\item In plaats van \verb$\define$ type \verb$\begin{define}[onderwerp] ..... \end{define}$: zie voorbeelden in dit \verb$.tex$ bestand
		\item Om een \verb$woord$ in de index te plaatsen, type \verb$\indx{woord}$ -- het verschijnt dan ``slanted'' in de \verb$.pdf$
		\item In \LaTeX\ zijn de braces \verb${...}$ de \LaTeX\ groep-open en \LaTeX\ groep-sluit operatoren -- en daarom niet zichtbaar in de tekst. Om ze zichtbaar te maken moet je \verb$\lbrace$ resp.{} \verb$\rbrace$ typen of -- minder slim als je \LaTeX\ beter kent type je \verb$\{$ resp.{} \verb$\}$.
		\item Als je zulke braces (\verb$\{\}$, \verb$()$, \verb$[]$) groot genoeg wilt hebben dan type je \verb$\left\lbrace ... \right\rbrace$:
		\[
		\left\lbrace \int_0^1 f(x) \text{d}x \right\rbrace.
		\]
		Een minder alternatief -- alleen te snappen met meer \LaTeX\ kennis is \verb$\big\{$ ipv.{} \verb$\left\rbrace$.
		\item Als je in het \verb$.pdf$ wilt kunnen klikken op secties om daarnaar toe te springen includeer dan
		\verb$\usepackage{hyperref}$ -- heb ik al gedaan -- dit package was al wel geincludeerd maar het automatisch springen was uitgeschakeld.
		\item De definities van \verb$\intr$ en \verb$\distr$ heb ik aangepast -- dat valt niet op maar daar komt later voordeel van
		\item De \verb$\AA$ geeft niet de gewenst ``interior'' notatie -- daarom heb ik \verb$\AA$ aangepast aan de \LaTeX\ standaard
		om $\mathring{A}$ te geven
		\item In plaats van \verb$\lim_{n\rightarrow\infty}$ type de kortere \LaTeX\ standaard \verb$\lim_{n\to\infty}$: $\lim_{n\to\infty}$
		\item De bold -- vetgedrukte -- vectoren veranderen in underline vectoren door 1 regel te wijzigen!
		\item In math mode -- tussen dollars en \verb$\[\]$ gebruik \verb$\ldots$ voor enumeraties zoals \verb$1,2,\ldots,d$: $1,2,\ldots,d$ -- gebruik \verb$\dots$ alleen in text mode.
	\end{itemize}
	Ik stel voor dat jullie deze tips verhuizen naar de appendix voor de volgende generatie -- die appendix heb ik toegevoegd.
	
	\ \\
	Groet,
	
	Jos.
\end{document}